% This uses the glossaries package. With this package, you can include multiple types of lists, track page numbers if desired, and define new lists. More information can be found at the following sites: https://mirrors.mit.edu/CTAN/macros/latex/contrib/glossaries/glossariesbegin.pdf and https://mirrors.rit.edu/CTAN/macros/latex/contrib/glossaries/glossaries-user.pdf https://www.overleaf.com/learn/latex/Glossaries


%*****************************************
%Define your list of glossary items below. Remember that the entries that you enter in this file will not automatically appear in the List of Symbols. You also have to reference the symbol in the body of your thesis by using the \gls command. 

%symbols
\newglossaryentry{deg}{name=$^\circ$, description={Degree}}
\newglossaryentry{grav}{name={1D}, description={Normal gravity environment}}
\newglossaryentry{wf}{name={\textit{f}}, description={Wear factor}}
\newglossaryentry{alp}{name={$\alpha$},description={Alpha}}
\newglossaryentry{theta}{name={$r_O$}, description={ecosystem respiration at reference temperature $T_a=0{^\circ}$C}}
\newglossaryentry{te}{name={$\tau_e$}, description={precision of the normal distribution of the likelihood}}
\newglossaryentry{q10}{name={$Q_{10}$}, description={multiplication factor to respiration with 10$^\circ$C increases in $T_a$}}
\newglossaryentry{phi}{name={$\phi$}, description={vapour pressure deficit response function}}
\newglossaryentry{del}{name=$\delta$, description={Transition coefficient constant for the design of linear-phase FIR filters which are used to take up space when testing the list of symbols}}

