% The top of your abstract will fill out automatically once you fill in the required fields on the main.tex file. In this file, you will provide your abstract body. Type your abstract body at the bottom of this page directly below the \doublespacing command.

\chapter{\texorpdfstring{\MakeUppercase{Abstract}}{Abstract}}
     \begin{center}
        \large
        \singlespacing
        \textbf{\thesistitle}\\
        \vspace{0.5cm}
        \large
        \textbf{\studentname}\\
        \vspace{0.5cm}
        \normalsize
        \ifdefined\thesis
        \textbf{A thesis submitted in partial fulfillment of the requirements \\for the degree of \degree}\\  
        \else
        \ifdefined\dissertation
        \textbf{A dissertation submitted in partial fulfillment of the requirements \\for the degree of \degree}\\ 
        \else
        \textbf{Please identify this document as either a thesis or dissertation on the main.tex in the section at the top that must be filled out.}\\
    \fi
    \fi
        \vspace{1cm}
        \textbf{\department}
        
        \vspace{0.25cm}

        \ifdefined\jointuni
        \textbf{The University of Alabama in Huntsville and  \jointuni}
        \else
        \textbf{The University of Alabama in Huntsville}
    \fi

        
        \vspace{0.1cm}
        \textbf{\gradmonth\ \gradyear}
        


    \end{center}
\vspace{0.1cm}

%****************************************************
%Enter the body of your abstract below. Remember there is a 150 word limit!
%****************************************************
\doublespacing
Throughout the 500 million year evolution of land plants, heterospory has evolved many times, and appears to be correlated with drop in chromosome numbers. This correlation was first observed over 50 years ago, but no hypothesis has survived experimental testing. Given the recent developments in computational analysis combined with published genome data across the plant tree of life, my goal is to explore the genomic changes associated with heterospory in a phylogenetic context. At its core, I tested the hypothesis that shifts to  heterospory probably share similar quantitative and qualitative signatures of molecular evolution. My goal is see if there are clues to how such molecular changes in gene families could potentially affect both spore production pattern as well as the mechanisms of meiosis that influence chromosome number.
\clearpage

