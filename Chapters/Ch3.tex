\chapter{Chapter 3. Conclusions and Future Work}

While organization is flexible, all theses and dissertations, no matter the discipline, share certain scholarly elements. You must provide an introductory statement or overview of your project; identify the significance of your investigation; discuss relevant literature to position your work; describe your methodology; state findings or results and their implications; and present conclusions and, if appropriate, recommendations for future work. Your chapters might be organized by kinds of information (for example, a literature review, methodology, and results), or you may organize conceptually with these elements logically interwoven.

Below is just an example table. Notice that captions for tables are placed above the table while captions for figures are placed beneath the figure. LaTeX automatically formats this correctly
\begin{table}[h]

\caption[Frequencies for equal-tempered scale, $A_4=440$]{Frequencies for equal-tempered scale, $A_4=440$ Hz. This table shows only the first five notes of a chromatic scale starting on $C_0$} %Just provide the title of the table in the square brackets. Then, in the next set of brackets, provide the entire caption (including the title again). By doing this, only the title of the table will be on the List of Tables instead of the entire caption. The first sentence of the caption can be the title.

\begin{center}
\begin{tabular}{|c | c | c |} 
 \hline
\textbf{Note} & \textbf{Frequency (Hz)} & \textbf{Wavelength} \\ [0.5ex] 
 \hline
 $C_0$ & 16.35 & 2109.89 \\ 
 \hline
 $C^{\#}_0/D^b_0$ & 17.32 & 1991.47 \\
 \hline
 $D_0$ & 18.35 & 1879.69 \\
 \hline
 $D^{\#}_0/E^b_0$ & 19.45 & 1774.20 \\
 \hline
 $E_0$ & 20.60 & 1674.62 \\ [1ex] 
 \hline
\end{tabular}
\end{center}


\end{table}

