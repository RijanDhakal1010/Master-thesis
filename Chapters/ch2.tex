\chapter{Chapter 2. Adding New Chapter, Creating Sections or Subsections, and Formatting Equations}

\section{Adding New Chapters}
Despite the non-trivial nature of the transition to heterospory and the changes in chromosome numbers, the mechanisms that underlie iterative transitions to heterospory have remained ambiguous (Kinosian et al. 2022). Why are changes in modes of reproduction and chromosome numbers associated with each other? Are there parallels between the genetic factors such as expansion and contractions of CNVs, or selection on specific genes that underlie the transitions?Are the CNVs and selection in the genes similar, opposite, or completely unrelated?
Convergent evolution in gene families within heterosporous and homosporous lineages could reflect the repeated transitions to heterospory in land plant lineages. Such changes may include expansions or contractions of gene family size or changes in the rates of nucleotide substitution that reflect selection on specific gene family members. I hypothesize that these gene families undergoing potential positive selection and copy number variation shared across multiple origins could be suggestive of neofunctionalization or subfunctionalization that led to heterospory. Models of nucleotide substitution rates may detect trends in the selection of specific gene families that underlie transitions to heterospory, or  other genomic changes undergoing similar selection rates could be behind the recurrent evolution of heterosporous plants.
Associating changes in gene copy number or selection with these transitions will not explain the causation behind the transitions to heterospory; that is a task for gain-of-phenotype research. However, it will improve our capacity to circumscribe more specific hypotheses to test for potential causes behind the correlation between the transition to heterospory and a reduction in chromosome number. As more homosporous species become genetically transformable, the candidate genes and gene families I am searching for will make excellent candidate genes for gene editing and functional validation.  
\section{Creating Sections or Subsections}
\subsection{Formatting Equations}

\begin{equation}
  \label{example}
  \begin{split}
   \nabla \cdot \nabla \psi &= \frac{\partial^2 \psi}{\partial x^2} + \frac{\partial^2 \psi}{\partial y^2} + \frac{\partial^2 \psi}{\partial z^2} \\
   &= \frac{1}{r^2 \sin\theta} \left[ \sin\theta \left( r^2 \frac{\partial \psi}{\partial r} \right) + \frac{\partial}{\partial \theta} \left( \sin \theta  \frac{\partial \psi}{\partial r} \right) + \frac{1}{\sin \theta} \frac{\partial^2 \psi}{\partial \varphi^2}  \right] 
     \end{split}
\end{equation}
Equation \ref{example} will hopefully help you understand how to properly format and reference equations in your document.

\subsection{Citations}
When you make your citations, you will need to first add them to the ref.bib file. Then, use the citation command followed by the name of the citation.\cite{Example:1} LaTeX allows you to control the style of your citations.\cite{Example:2} On the main.tex file, set your bibliography style to the one you prefer. 



