
\chapter{Chapter 1. Introduction}%Be sure to include Chapter 1. before you write the name of your chapter. Name all remaining chapters in the same manner.

\section{Formatting}

In animals, egg and sperm cells form by meiosis, whereas this occurs via mitosis in plants. Plants have an alternation of multicellular diploid and haploid phases. The haploid phase arises from a spore, which in plants are the product of meiosis. Spores can either be the same size (homospory) or two distinct sizes (heterospory). In heterosporous species, the smaller microspore germinates to form a multicellular male gametophyte, which produces a sperm cell through mitosis; the larger megaspore germinates into a multicellular female gametophyte that produces an egg cell (Figure 1). Spores of homosporous species germinate and produce bisexual gametophytes, able to bear both egg and sperm on the same individual. Albeit, it is not uncommon for homosporous plants such as mosses to have separate sexes.

Extant heterosporous plants consist of three lineages: heterosporous ferns, all seed plants, and heterosporous lycophytes. Most plant species are heterosporous angiosperms (flowering plants), whereas the most common homosporous species are bryophytes, ferns, and lycophytes (the club mosses). All other land plants, including homosporous ferns and lycophytes, are homosporous. There have been at least 11 independent transitions to heterospory from the ancestral condition of homospory in vascular plants, but only three of these transitions are extant (Figure 2). The repeated evolution of heterospory represents convergence in vascular plant lineages that do not share a most recent common ancestor (Bateman and DiMichele 1994). The fundamentally different modes of reproduction between the two types of plants make the transition from homospory to heterospory a non-trivial one, and the evolution of heterospory has been labeled as the most significant iterative innovation in the evolution of vascular plants (Bateman and DiMichele 1994).

The difference in chromosome numbers between homosporous and heterosporous plants has been investigated for the past 55 years, but inferred relationships remain ambiguous (Kinosian, Rowe, and Wolf 2022). However, there is evidence that significant changes in base chromosome numbers have accompanied the independent evolution of heterospory (Figure 2). On average, ferns have n= 57 chromosomes, while the mean angiosperm chromosome number is n = 13 (Klekowski and Baker 1966). A more recent meta-analysis of plant chromosome counts substantiated the previous analysis, and the significant differences between heterosporous and homosporous plants remain, with means of 2n = 115 for homosporous plants and 2n = 27.24 for heterosporous plants (Figure 3). In some cases, homosporous species can have incredibly high chromosome numbers; for instance, the homosporous fern Ophioglossum reticulatum (commonly known as the “Adder’s tongue fern”) has 1260 chromosomes (Patel and Reddy 2018), more than any other known eukaryote. 

 Earlier approaches to explaining the difference in chromosome numbers between heterosporous and homosporous plants focused on understanding why homosporous plants accumulate an increasing number of chromosomes over time (Kinosian, Rowe, and Wolf 2022). The once-dominant theory was that homosporous plants primarily reproduce via gametophytic selfing, the fusion of gametes produced by mitosis from the same gametophyte (parent). Gametophytic selfing produces completely homozygous zygotes/offspring and would necessitate polyploidy-based redundancy to avoid genetic load, therefore leading to selection for larger genomes (Hickok 1978). However, the tendency of polyploids to act as genetic diploids disproved the once prominent gametophytic selfing hypothesis (C. H. Haufler and Soltis 1986). The rejection of the gametophytic selfing-based hypothesis presented by Klekowski (Christopher H. Haufler 2014) coincided with a shift from morphology-based exploration to molecular-based studies of plant phylogenetics and evolution.
As phylogenetic research of homosporous plants began to incorporate genomic methods, information from gene copy sequence similarity patterns indicated that homosporous plants have had lower rates of paleopolyploidy than heterosporous plants despite having more significant chromosome numbers today. Instead, compared to angiosperms, high chromosome numbers in homosporous plants seem to result from higher retention of chromosomes from the fewer rounds of polyploidy (Barker 2009). Higher retention, rather than expansion, suggests that homosporous lineages are not outliers that accumulate chromosomes faster than non-homosporous lineages. Instead, it suggests that heterosporous lineages have perhaps undergone higher rates of paleopolyploidy and genome downsizing via reduction in chromosome numbers (Barker 2009; Clark et al. 2016; Li et al. 2021; Liu et al. 2019; Wang et al. 2021; Carins Murphy, Jordan, and Brodribb 2017)

\begin{figure}[ht]
    \centering
    \includegraphics[scale=.7]{Figures/figure 1.1.jpg}
    \caption[11 Most Common Grammar Mistakes Employees Make: I'm purposely making this longer to extend to two lines.]{11 Most Common Grammar Mistakes Employees Make. When labeling your figures, single-space if captions extend to two lines}
    \label{fig 1.1}
\end{figure}

\section{Symbols}

If your document includes many symbols or acronyms, you may include a List of Symbols, Abbreviations, \textit{etc}. If you want a symbol/abbreviation included in the List of Symbols, be sure to create an entry for it first on the List of Symbols Glossaries.tex file. Once it is created, then you can insert it with a glossaries command. For example, the current temperature outside is 100\glspl{deg}.

You can capitalize your symbols or make them plural by using different commands included with the glossaries package. However, only those symbols that are actually referenced in the body of your thesis will be present in the List of Symbols. Below are a few more symbol examples.

\gls{grav}

\gls{wf}

\gls{alp}

\gls{theta}

\gls{te}

\gls{q10}

\gls{phi}

