\chapter{Chapter 6. Discussion}

\section{Copy number variation analysis}

It has been over 50 years since Klekowski and Baker (1966) first pointed out the
association between high chromosome numbers in homosporous plants. In a phylogenetic
context, it now appears that chromosome number reduction has occurred on the lineages leading
to heterospory. However, the scientific community still lacks a plausible explanation for this
association of two seemingly unrelated traits. The research goal was to look for signatures of
evolution that could provide hints about gene families that are playing a significant role in the
evolutionary dynamic of the correlation between heterosporous spore production. Signatures of
evolution from gene family expansion or contraction analysis reveal a limited number of parallel
changes on heterosporous lineages. There were no gene families with significant changes leading
to all three nodes. For pairwise intersections, there was one gene with significant changes leading
to heterosporous lycophytes and heterosporous ferns and one gene family that had significant
changes leading to heterosporous lycophytes and seed plants. There were no intersections
between seed plants and heterosporous ferns.
OG0001580 is the one gene family that was shared between heterosporous ferns and
heterosporous lycophyta. The Arabidopsis thaliana genes present in this homologous family are
members of the homeobox gene family. The homeobox genes are known to have large effects in
early morphological development and mutations can lead to some severe homeosis \cite{Gehring1993-mf}. OG0000619 was the one gene family that was shared between seed plants and
heterosporous lycophyta. The Arabidopsis thaliana genes present in this homologous family are
known as Hothead genes responsible for floral structure and do not follow Mendelian
segregation. Hothead genes have roles in determining floral morphology- something closely
related to spore production, and Hothead genes in Arabidopsis thaliana have been observed to
undergo non-Mendelian inheritance \cite{Rhee2003-ww} which could trace its causes to meiosis.
There is some history of research into the non-Mendelian nature of these specific genes archived
in The Arabidopsis Information Resource (TAIR). Future work should explore these genes and
the factors involved in more detail.
Objectively, the genes of this gene family are homeobox genes and hothead genes only in
Arabidopsis thaliana, and could have completely different molecular and biological roles in
other species. Albeit the significant expansion of genes on two lineages leading to heterosporous
warrant further investigation into their biological and molecular function across species.

\section{Selection Analysis}

The study of trait evolution across families of green plants using natural selection data in a
phylogenetic context requires access to more computational power to generate a greater density
of data than is currently available. This should not be an issue given the steady progress in data
generation being made by the computational systems this project has access to.

\section{Limitations of the study}

It is possible for the gene family reconstruction or the associated functions of the gene
families by CAFE5 to be spurious and entirely random. I used CAFE5 to generate data for
thousands of families on a large species tree with a deep phylogeny. However, this sample set
was pushing the limits of the algorithms implemented within CAFE5. Based on direct
communications with the authors of CAFE5, apparently, the tree was “too large and too deep”.
For the reconstruction of gene families, for the 9461 families that were successfully
reconstructed, 40 of the largest families could not be compiled. Those 40 homologous families
accounted for some tens of thousands of genes, and while there is no promise of good data, it
was a non-negligible portion of the total data that simply could not be processed. The difficulty
CAFE5 had in reconstructing gene families, given the size and the depth of the phylogeny,
reduces confidence in the output and warrants some further investigation with some alternative
approaches.

Some solutions to the size and depth issues that can simultaneously provide some hints
about spuriousness in the results:

\begin{itemize}

    \item Compromise the generous phylogenetic background data and use a smaller and shallower
    species tree. The tree can be made shallower by giving up outgroups but it is currently
    unclear how effective that can be.

    \item Use multiple lists of taxa with analogous phylogenetic outlines and compare the
    outcomes between each other. The comparisons between outputs neither exactly provide
    specificity nor exactly provide specificity, but if there is a difference at all, then it brings
    the status of the reconstruction into question.

    \item Use parsimony based approaches for gene family reconstruction as opposed to maximum
    likelihood estimation altogether. Our implementation of MLE in this project was thanks
    to the robust history of MLE’s capacity to generate data on smaller datasets.
\end{itemize}

Depending on quality, the transcriptome of a species can be an incomplete representation
of the genes present in that species. One metric of transcriptome quality is BUSCO \cite{Simao2015-xq}, which is a method to gauge the quality of a genome using single copy orthologous genes,
where the completeness of a genome can be gauged by comparison against a standardized
database. BUSCO produces a percentage-based quality score that reflects the completeness of
any specific sample of sequencing data. The percentage scores reflect the completeness of the
sample sequencing data. While the taxon list was fairly exhaustive and representative of the
phylogenetic context, the BUSCO scores for the samples ranged from the low 70s to the high 90s
in terms of percentage. It is possible that, within that range of transcriptome quality scores across
the list of taxon, some of the gene families as found by my runs of homology inference, do not
reflect copy number variation in the gene families with complete accuracy. The range of quality
of the transcriptomes has a potential to affect CNV and by extension the reconstruction families
across the phylogeny. We can only gauge the actual size of the impact from the range of BUSCO
scores with a completely different list of samples. A “simple” solution to this is to use taxa where
all species have BUSCO scores in the high 90s.
I built the research in this project around exploring a potential relationship between spore
production and chromosome numbers, but these two traits could be unrelated. A way to gauge
the existence of a potential relationship between spore production and chromosome numbers
would be to expand the current experiment design to include heterosporous nodes alongside
other nodes where there has been a marked decrease in chromosome numbers without the
evolution of heterospory.
The study of homology from the perspective of gene families/orthogroups provides an
objective avenue to gauge the similarities and differences between species, families, lineages and
nodes. This comes back to the fact that gene families can be identified, labeled/tagged and easily
parsed for insights. While objective and easy to implement, looking at phylogenetics with
well-set gene families misses the possibility of catching similarities in functions shared across or
between two or more than two different gene families. In other words, it is easy to parse
information tagged using gene families, but parsing information by the function of gene families
across different gene families requires a different approach, which so far does not have a
standardized approach. While a standardized approach is lacking, manual parsing of the results
in this project is possible, given the small number of gene families to be investigated. As such, a
manual investigation of the significantly expanding gene families will be a part of the expansion
of this project.
The research objectives, built around the hypothesis that the evolution of correlated traits
in heterosporous lineages probably share a similar history of evolution, were to search for
similarity in signatures of evolution. Within this sample set and methods, while significantly
smaller than initially expected, there were two gene families that warrant further investigations
into their potential role in floral morphology/spore production and related roles in meiosis. When
I started this project, there were no phylogenetic and genomic insights on the evolution of
heterospory and its correlation with drops in chromosome numbers from the perspective of gene
families. The exploratory nature of this study means that the insights collected here require
further investigation with the use of functional genomics. Nevertheless, this project has provided
some novel insights that make it possible to approach functional genomics at all.